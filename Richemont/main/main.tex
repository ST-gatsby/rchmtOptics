\documentclass[a4paper,12pt]{article}

%=========== PACKAGES ================

\usepackage[french]{babel}
\usepackage[utf8x]{inputenc}
\setlength{\headheight}{20.14264pt}
\addtolength{\topmargin}{-8.14264pt}
%\usepackage{subfigure}
%\usepackage{subcaption}
% pour gérer les positionnement d'images
\usepackage{float}
\usepackage{amsmath}
\usepackage{graphicx}
\graphicspath{{../Pictures/}}
\usepackage{wrapfig}
\usepackage[justification=centering]{caption}
%\usepackage{enumitem}
\usepackage[colorinlistoftodos]{todonotes}
\usepackage{url}

%pour les informations sur un document compilé en PDF et les liens externes / internes
\usepackage{hyperref}
\hypersetup{
    colorlinks=true,
    linkcolor=blue,
    filecolor=magenta,      
    urlcolor=cyan,
    pdftitle={Overleaf Example},
    pdfpagemode=FullScreen,
    }
%pour la mise en page des tableaux
\usepackage{amsmath}
\usepackage{array}
\usepackage{tabularx}
\usepackage[final]{pdfpages} 
\usepackage{placeins}
%pour utiliser \floatbarrier
%espacement entre les lignes
\usepackage{setspace}
%modifier la mise en page de l'abstract
%\usepackage{abstract}
%police et mise en page (marges) du document
\usepackage[T1]{fontenc}
\usepackage[top=3cm, bottom=3cm, left=3cm, right=3cm]{geometry}
%Pour les galerie d'images
\usepackage{subfig}
%\usepackage{titlesec}
\usepackage{sectsty}
\usepackage{tikz}
\usepackage{pgfplots}
\usepackage{fancyhdr}
\usepackage{stmaryrd}
\usepackage{amssymb}
\usepackage{calrsfs}
\DeclareMathOperator{\e}{e}
\parskip=3pt


%======= INFORMATION ET REGLES ==============

\hypersetup{
% Information sur le document
pdfauthor = {Gabriel Gostiaux},	% Auteurs
pdftitle = {},   % Titre du document
pdfsubject = {},		    % Sujet
pdfstartview={FitH}}% ajuste la page à la largeur de l'écran

%======= EN-TETE ET PIED DE PAGE ==========

\pagestyle{fancy}
\fancyhead[L]{Etude Richemont : Métaux tranlucides}
\fancyhead[R]{\includegraphics[width=0.2\textwidth]{Logo-IOGS.png}}
\renewcommand{\headrulewidth}{1pt}
\newcommand{\HRule}{\rule{\linewidth}{0.5mm}}
\renewcommand{\footrulewidth}{1pt}
\fancyfoot[L]{Gostiaux Gabriel}
\fancyfoot[C]{}
\fancyfoot[R]{\thepage}

%======== DEBUT DU DOCUMENT ==========

\begin{document}

\begin{titlepage}
\begin{center}

% Upper part of the page. The '~' is needed because only works if a paragraph has started.
\includegraphics[width=0.75\textwidth]{Logo-IOGS.png}~\\[1.5cm]

%\textsc{\LARGE Ecole Centrale de Lyon}\\[1.5cm]

\textsc{\Large }\\[0.5cm]

% Title
\HRule \\[0.4cm]

{\bfseries
\huge \textsc{Richemont : le métal transparent}\\[0.5cm]
\Large  TP Projets S7}

\HRule \\[1cm]

% Author and supervisor
\begin{minipage}{0.8\textwidth}
\begin{flushleft} \large
%\emph{Groupe 4, binôme 7 :}\\[0.4 cm]
\textsc{Gostiaux} Gabriel \\[1cm]
\textsc{Diop} Sidy \\[1cm]
\end{flushleft}
\end{minipage}


\textsc{\Large }\\[1cm]
Nous attestons que ce travail est original, que nous citons en référence toutes les sources utilisées et qu’il ne comporte pas de plagiat.

\textsc{\Large }\\[1cm]
%\includegraphics[width=0.5\textwidth]{Figures/}




\vfill

% Bottom of the page
{\large \today}

\end{center}
\end{titlepage}
\tableofcontents
\clearpage
Principles of Optics Max Born and Emil Wolf 7th edition.

We use the formulas of reflection and transmission : our thickness range is $[30, 90]$ nanomètres. At normal incidence, supposing that $exp(2v_2\eta)\geq 50$ which correspond to $h\geq 25 \; nm$ the following simplification involve an error of $2\%$ :

\begin{equation}
  \mathcal{R} \simeq \rho_{12}^2 \quad \mathcal{T} = \frac{n_3cos(\theta_3)}{n_1cos(\theta_1)}\tau_{12}^2\tau_{23}^2e^{-4v_2\eta}
\end{equation}

with 
\begin{align*}
  2u_2^2 &= \left[n_2^2(1-\kappa_2^2)-n_1^2sin^2(\theta_1)\right] + \sqrt{\left[n_2^2(1-\kappa_2^2)-n_1^2sin^2(\theta_1)\right]^2 + 4n_2^4\kappa_2^2}\\
  2v_2^2 &= -\left[n_2^2(1-\kappa_2^2)-n_1^2sin^2(\theta_1)\right] + \sqrt{\left[n_2^2(1-\kappa_2^2)-n_1^2sin^2(\theta_1)\right]^2 + 4n_2^4\kappa_2^2}
\end{align*}

and 
\begin{align*}
  \eta &= \frac{2\pi}{\lambda_0}h\\
  \rho_{12}^2 &= \frac{(n_1cos(\theta_1) - u_2)^2+v^2}{(n_1cos(\theta_1) + u_2)^2+v^2}\\
  \tau_{12}^2 &= \frac{(2n_1cos(\theta_1))^2}{(n_1cos(\theta_1) + u_2)^2+v^2}\\
  \tau_{23}^2 &= \frac{4(u_2^2+v_2^2)}{(n_3cos(\theta_3) + u_2)^2+v^2}
\end{align*}

The formulas (1) depends only of refractive index, angles, wavelength, extinction coefficient and thickness. Wavelength and extinction coefficient are bound to each other so we must take that into account. In our case, we work at normal incidence so we will suppose $\theta_1=\theta_3=0$, the top dielectric environmet is composed of air, and the bottom one is composed of glass, so we will suppose $n_1=1, \; n_3=1,52$. We will think about the substract material later. Indeed, we would like to use quartz, but, since it is a birefringent material this complicate things. This leads to

\begin{equation*}
  u_2 = n_2 \quad v_2 = n_2 \kappa_2
\end{equation*}

\begin{align*}
  \rho_{12}^2 &= \frac{(1-n_2)^2 + n_2^2\kappa_2^2}{(1+n_2)^2 + n_2^2\kappa_2^2}\\
  \tau_{12}^2 &= \frac{4}{(1+n_2)^2 + n_2^2\kappa_2^2}\\
  \tau_{23}^2 &= \frac{4(n_2^2+n_2^2\kappa_2^2)}{(1+n_2)^2 + n_2^2\kappa_2^2}
\end{align*}

d'où

\begin{align}
  \mathcal{R} &\simeq \frac{(1-n_2)^2 + n_2^2\kappa_2^2}{(1+n_2)^2 + n_2^2\kappa_2^2}\\
  \mathcal{T} &= 1,5\times\frac{16\;n_2^2(1+\kappa_2^2)}{\left[(1,5+n_2^2)^2 + n_2^2\kappa_2^2\right]^2}e^{-8\pi n_2^2 \kappa_2^2 \frac{h}{\lambda}}
\end{align}

Considering the evolution of the refractive index and of the extinction coefficient of silver, the numerical evaluation can be given in an array-shape :
\listoffigures
\end{document}
